\documentclass{cv}

\begin{document}

\phantom{}

\vspace{1.5cm}

To

\textbf{\textsc{<Recruiter Name>}} \\*
<Recruiter Position>               \\*
<Company>                          \\*
<Address>

\vspace{2cm}

Dear <Recruiter Name>,

\vspace{1cm}

PARAGRAPH ONE: State reason for letter, name the position
or type of work you are applying for and identify source
from which you learned of the opening. (i.e. web pages, newspaper,
employment service, personal contact).

\vspace{0.5cm}

PARAGRAPH TWO: Indicate why you are interested in the position,
the company, its products, services - above all, stress what you
can do for the employer. If you are a recent graduate, explain
how your academic background makes you a qualified candidate for
the position. If you have practical work experience, point out
specific achievements or unique qualifications. Try not to repeat
the same information the reader will find in the resume. Refer
the reader to the enclosed resume or application which summarises
your qualifications, training, and experiences. The purpose of this
section is to strengthen your resume by providing details which
bring your experiences to life.

\vspace{0.5cm}

PARAGRAPH THREE: Request a personal interview and indicate your
flexibility as to the time and place. Repeat your phone number
in the letter and offer assistance to help in a speedy response.
For example, state that you will be in the city where the company
is located on a certain date and would like to set up an interview.
Or, state that you will call on a certain date to set up an interview.
End the letter by thanking the employer for taking time to consider
your credentials.

\vspace{1.0cm}

Sincerely,

\vspace{0.2cm}

<Recruitee Name>

\vspace{1.0cm}

\faPaperclip \hspace{0.5em} Curriculum Vitae

\footer{Cover Letter\quad $\cdot$ \quad <Recruitee Name>}

\end{document}
